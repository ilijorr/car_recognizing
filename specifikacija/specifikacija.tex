\documentclass{article}
\usepackage[T2A]{fontenc}
\usepackage[utf8]{inputenc}
\usepackage{cmap}
\usepackage{hyperref}
\usepackage{dirtree}
\usepackage{geometry}
\usepackage{listings}
\usepackage{xcolor}

\definecolor{codegreen}{rgb}{0,0.6,0}
\definecolor{codegray}{rgb}{0.5,0.5,0.5}
\definecolor{codepurple}{rgb}{0.58,0,0.82}
\definecolor{backcolour}{rgb}{0.95,0.95,0.92}

\lstset{
    backgroundcolor=\color{backcolour},   
    commentstyle=\color{codegreen},
    keywordstyle=\color{magenta},
    numberstyle=\tiny\color{codegray},
    stringstyle=\color{codepurple},
    basicstyle=\ttfamily\footnotesize,
    breakatwhitespace=false,         
    breaklines=true,                 
    captionpos=b,                    
    keepspaces=true,                 
    numbers=left,                    
    numbersep=5pt,                  
    showspaces=false,                
    showstringspaces=false,
    showtabs=false,                  
    tabsize=2
}

\geometry{
  a4paper,           
  left=15mm,        
  right=15mm,      
  top=20mm,       
  bottom=20mm,   
}

\title{%
  Софтвер за препознавање модела аутомобила \\
  \large Спецификација пројекта из предмета \\
  основе рачунарске интелигенције}
\author{Илија Јордановски, SV 73/2022}
\date{Нови Сад, Август 2025.}

\begin{document}
  \pagenumbering{gobble}
  \maketitle
  \newpage
  \pagenumbering{arabic}

  \section{Дефиниција проблема}
  Препознавање произвођача, модела и годишта аутомобила из дигиталних фотографија представља проблем компјутерске визије и машинског учења.
  Може се дефинисати као вишекласна класификација са више излаза, где је улаз дигитална фотографија аутомобила, а излаз су три засебне предикције: 
  \begin{enumerate}
    \item \textbf{Произвођач аутомобила}(Волксваген, Мерцедес, Фиат...)
    \item \textbf{Модел аутомобила} (Голф, Пунто...)
    \item \textbf{Година производње}
  \end{enumerate}

  \section{Мотивација}
  Систем за аутоматско препознавање возила има широку примену у различитим секторима:
  \subsection*{Превенција преваре на наплатним рампама}
  На наплатним рампама систем ће аутоматски упоредити податке са регистарских таблица са визуелно \\
  препознатом марком и моделом возила. Уколико дође до неусоглашености, 
  систем генерише аларм за детаљнију проверу, спречавајући преваре са лажним таблицама.
  \subsection*{Анализа тржишта и конкуренције}
  Маркетиншки тимови аутомобилских компанија могу добити стварне податке о присуству различитих врста аутомобила на одређеним подручјима(градски паркинг, зоне града, пословне зграде). 
  Овакви подаци су много меродавнији од података из званичних статистика продаје.
  \subsection*{Апликације и сајтови за трговину колима}
  Овакав систем би корисницима сајтова као што су \textit{PolovniAutomobili} или \textit{mobile.de} олакшао претрагу у случају да нису сигурни који модел возила је у питању.
  Поред тога, при објављивању новог огласа се може упоредити резултат визуелно препознатог аутомобила са подацима из огласа. У случају непоклапања, администратор мора да одобри креирање огласа.

  \section{Скуп података}
  Скуп података који би служио за обуку и евалуацију модела је јавно доступан скуп података \emph{"Car Models"} са платформе \textit{HuggingFace}. \\
  Адреса која води до скупа података је: 

  {\centering
    \url{https://huggingface.co/datasets/natsu39/car-models}
  \par}

  \subsection*{Опис скупа података}
  Скуп података садржи фотографије 2966 различитих возила, направљених од стране 80 произвођача.

  \subsection*{Атрибути}
  \begin{enumerate}
    \item \textbf{Фотографија аутомобила} (улазни атрибут)
      \begin{itemize}
        \item \textbf{Тип:} РГБ фотографија
        \item \textbf{Формат:} JPG, различитих димензија, захтева нормализацију при обради
      \end{itemize}
    \item \textbf{Путања до фотографије} (метаподатак из којег се добијају обележја) 
      \begin{itemize}
        \item \textbf{Тип:} стринг
        \item \textbf{Формат:} \verb|.../MAKE_MODEL_YEAR/image_name.jpg|, парсира се како би се дошло до циљних обележја
      \end{itemize}
  \end{enumerate}
  \subsubsection*{Циљна обележја}
  \begin{itemize}
    \item \textbf{make} (стринг, категорички)
      \begin{itemize}
        \item \textbf{Број класа:} 80, где BMW, Mercedes Benz и Audi су најзаступљенији и садрже 7.1\%, 6.4\% и 6.1\% укупног узорка, редом. 
      \end{itemize}
   \item \textbf{model} (стринг, категорички)
    \begin{itemize}
      \item \textbf{Број класа:} 1196
      \item \textbf{Расподела:} Врло неуравнотежена. Медијан броја модела по произвођачу је 7, за модел се може рећи да има дугачак реп.
    \end{itemize}
    \item \textbf{year} (стринг, категорички)
    \begin{itemize}
      \item \textbf{Број класа:} 104 у опсегу од 1921. до 2024.
      \item \textbf{Расподела:} Преко 92\% скупа чине возила произведена од 2000. године надаље. Истиче се период 2010-2019 из којег потиче 50.57\% података.
    \end{itemize}
 \end{itemize}
  \section{Начин претпроцесирања података}
  Пре него што се проследе моделу на тренирање, подаци морају да се обраде на следећи начин:
  \begin{itemize}
    \item \textbf{Фотографије}
      \begin{enumerate}
        \item Фотографије се преувеличавају на димензије компатибилне са моделом(224x224 за ResNet-50).\\ 
          Преувеличавање се може извршити кроз исецање, допуну, развлачење фотографије и кроз њихову комбинацију.
        \item Након што је фотографија одговарајуће величине, на њу се примењују насумичне аугментације. Аугментације могу бити ротација, хоризонтално пресликавање, промена осветљења, замагљење, итд...
        \item Последњи корак обухвата претварање фотографије у тензор, који се након тога нормализује уз помоћ средње вредности и стандардне девијације модела тренираним \\ 
          над ImageNet скупом података([0.485, 0.456, 0.406] и [0.229, 0.224, 0.225], редом)
      \end{enumerate}
    \item \textbf{Обележја}

      Обележја се добијају обрадом путање до слике, тј. назива директоријума у којем се фотографија налази. \\ 
      Уколико се фотографија налази у директоријуму под називом \verb|'/AUDI_A3_2008/'|, обележја за фотографију су \verb|{make: "AUDI", model: "A3", year: "2008"}|
  \end{itemize}
  Након овакве обраде, подаци су компатибилни са моделом и могу се користити за даље тренирање.
  \section{Методологија}
 Главна идеја имплементације оваквог система јесте \textbf{усавршавање (fine-tuning)} већ постојећих модела \\ 
 заснованих на \textbf{ResNet-50} архитектури, претходно обучених на \textbf{ImageNet} скупу података. Овакав модел је одличан избор јер већ располаже основним знањем о визуелним карактеристикама које су применљиве и на домен аутомобила. 
 \subsection*{Кораци решавања проблема}
  Усавршавање се изводи кроз 3 главна корака:
  \begin{enumerate}
    \item \textbf{Додавање новог последњег слоја}
      Оригинални последњи потпуно повезани слој ResNet-50 модела (димензија 1000 излаза) замењује се са нова три потпуно повезана слоја. Сваки од ових слоја је задужен за предвиђање једне од циљних категорија
    \item \textbf{Тренирање са \textit{замрзнутим} слојевима}
      Овај корак обухвата тренирање модела при којем се искључиво мењају тежине последња 2 слоја неуронске мреже, од којих је једна додата у претходном кораку. Остали слојеви се "замрзавају" како се не би изгубило већ стечено опште знање модела. Овакво тренирање заустављамо када модел престане да напредује на валидационом скупу.
    \item \textbf{Тренирање свих слојева}
      Након што модел добије идеју о градијентима и тежинама за последње слојеве мреже, врши се тренирање модела код којег се мењају тежине свих слојева мреже. Овакво тренирање заустављамо када модел престане да напредује на валидационом скупу.
    \item \textbf{Упоредна анализа са EfficientNet архитектуром}\\
Како бисмо потврдили избор архитектуре и добили увид у перформансе система, имплементирамо и усавршавамо \textbf{EfficientNet-B3} модел, такође претходно обучен на ImageNet скупу. EfficientNet представља савремену конволуциону архитектуру оптимизовану за бољи однос тачности и вычислиonalне сложености. Овај модел се тренира и тестира истоветним протоколом као и ResNet-50 модел, што нам омогућава директну поређене перформанси и анализу предности и мана обе архитектуре за задати проблем.
  \end{enumerate}
  \subsection*{Технички детаљи}
  Тренирање ће се вршити у серијама од $32$ или $64$ слике. \\ 
  Стопа учења која ће се користити у првој фази тренирања треба да буде већа вредност ($\sim 0.001$), уз распоређивач који ће мењати стопу по потреби уз помоћ ReduceLROnPlateau принципа. \\ 
  У другој фази тренирања, стопа учења ће бити мања, нпр. ($\sim 0.0001$), уз распоређивач који користи косинусно спуштање.
  \subsection*{Улаз и излаз}
  \begin{itemize}
    \item \textbf{Улаз}: Модел прима слику аутомобила у облику тензора нормализованих пиксела.
    \item \textbf{Излаз}: Модел враћа три одвојена тензора вероватноћа:
    \begin{enumerate}
        \item Предвиђање марке (највероватнији индекс = име марке)
        \item Предвиђање модела (највероватнији индекс = име модела)
        \item Предвиђање годишта (највероватнији индекс = годиште)
    \end{enumerate}
  \end{itemize}
  \section{Начин евалуације}
  \subsection*{Подела података}
  С обзиром на то да скуп података долази без конкретне поделе, подаци се требају поделити на скуп за тренинг и тестирање, у размери 80-20 респективно. Након тога, скуп за тренинг се дели на подскупове за тренинг и евалуацију, у размери 80-20. \\ 
  Финална расподела је:
  \begin{itemize}
    \item \textbf{тренинг} -- 64\%
    \item \textbf{евалуација} -- 16\%
    \item \textbf{тестирање} -- 20\%
  \end{itemize}
  \subsection*{Метрика процене}
  Пошто је задатак вишекласна класификација са више излаза, метрике се морају пратити за сваки излаз посебно. \\
  Како расподела произвођача није уравнотежена, F-Score метрика је добар избор зато што представља хармонијску средину између прецизности и одзива. \\ 
  Што се годишта тиче, RMSE је бољи зато што кажњава велике грешке, и прикладнији је од Ф1 за ретке класе. \\
  Метрика за процене на нивоу целог система је EMR, зато што гарантује тачност свих излаза истовремено.
  \section{Технологије}
  Систем може бити реализован уз помоћ језика Python, радних оквира PyTorch, torchvision и fastai.
  \section{Литература}
  Пример сличног готовог производа се налази на \url{https://carnet.ai/}. \\
  PyTorch документација \url{https://docs.pytorch.org/docs/stable/index.html}. \\
  fastAi документација \url{https://course.fastai.com/}. \\
  torchvision документација \url{https://docs.pytorch.org/vision/stable/index.html}. \\
\end{document}
